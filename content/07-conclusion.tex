%!TEX root =  ../final-report.tex

\chapter[Conclusions]{Conclusions}
\label{sec:conclusions}

The purpose of this project was to design a portable and affordable system for detecting moisture inside a grain bin using microwave imaging techniques. There were several different components comprising this system. Two types of antennas were studied; H-field and E-field antennas. A VNA was required for sending and receiving signals to and from the antennas and a multiplexer was required in order to connect the VNA to the array of antennas and perform the necessary switching. Also, a microprocessor and software was needed to provide control to the multiplexer and perform data acquisition and management.

It was found that the E-field antennas designed were not suitable for real world use for this application however the H-field antennas performed well. Due to time constraints and issues with our original supplier for our PCBs we were unable to complete the multiplexer in time for this report, however we hope to have something in time for our presentation. For testing our system we used an existing switch provided by EIL. The miniVNA PRO was a very affordable option but we were unable to get direct access to data obtained through it which resulted in our system being very slow in running through its data acquisition procedure. If direct access to this data were possible or a different VNA was used which allowed this direct access, our system would perform much quicker.

With a bit more work to complete with the multiplexer and resolving the issue with the limitation of the MVP's manufacturer software, we feel that we can succeed at creating an affordable and portable system for detecting moisture inside a grain bin.
%!TEX root =  ../final-report.tex

\chapter[Introduction]{Introduction}
\label{sec:intro}


Our project is aimed at researching into and developing a better topology for the acquisition of s-parameters data from a grain bin. These parameters of the bin’s contents can then be analyzed to produce an image that displays the grains dielectric permittivity properties to detect water contamination. To accomplish our goal, the project was broken into a research phase, a design and integration phase and a calibration and field testing phase. These phases would be responsible for a well rectified integration between sixteen E-field and H-field antennas, a DC/RF multiplexer switch box, two microcontroller chips and a portable vector network analyzer (VNA).

The array of 16 antennas, consisting of both E-field and H-field antennas, will be built and installed in a full size grain bin. A 2-port VNA, via the use of a microcontroller, will transmit a signal though a microcontroller-controlled DC/RF multiplexer switch box to a single antenna and receive the signal reflected back through the other antennas. The microcontroller will then collect and format the received data such that it can be processed later to create an image of the grain bin contents on an external PC.

The research phase was completed over the first two month of our project. It involved investigation into different antenna designs, study of multiple RF switch ICs, inquiry into cheap and portable VNAs and microcontrollers suitable for our purposes, and exploration of various Electrostatic Discharge (ESD) protection modules. 

The design and integration phase, which we are currently in, involves antenna building, switch box RF/DC PCB design, microcontroller programming, VNA calibration and finally an integration of all these components into one single unit with a portable power source. We would be finished this phase by the end of the month

The final calibration and field testing phase, involves making tweaks to the hardware units so it all fits together and works as intended. We would also be testing it out in a full-sized grain bin. This is anticipated to be completed over the month of February. 





